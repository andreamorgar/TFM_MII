\chapter*{}
%\thispagestyle{empty}
%\cleardoublepage

%\thispagestyle{empty}

\input{portada/portada_2}



\cleardoublepage
\thispagestyle{empty}

\begin{center}
{\large\bfseries Aprendizaje predictivo en Nutrición}\\
\end{center}
\begin{center}
Andrea Morales Garzón\\
\end{center}

%\vspace{0.7cm}
\noindent{\textbf{Palabras clave}: Word Embedding, Food Computing, Alineamiento de datos, Lógica Difusa, Procesamiento de Lenguaje Natural}\\

\vspace{0.7cm}
\noindent{\textbf{Resumen}}\\

La integración de datos heterogéneos es una tarea indispensable en múltiples dominios y, en especial, en el área de Food Computing. Los grandes volúmenes de datos que intervienen, el vocabulario hiperespecializado, los factores multiculturales y las asunciones en el lenguaje alimenticio la convierten en una tarea difícil de abordar. En este trabajo se propone un método general basado en modelos predictivos del lenguaje para alinear fuentes de datos a partir de las descripciones textuales de sus elementos. En concreto, se ha desarrollado un sistema que combina un modelo de tipo \textit{Word Embedding} para representar los datos textuales y un procedimiento de mapeo entre ítems basado en medidas de distancia sintácticas y semánticas, tanto clásicas y difusas.
Con el fin de reflejar sus capacidades y su versatilidad, el sistema se ha aplicado para resolver un problema de interés actual: la adaptación automatizada de recetas a restricciones alimenticias. Las pruebas empíricas realizadas muestran que esta aproximación es apropiada para resolver el problema, especialmente cuando se combinan modelos semánticos de dominio específico con medidas de distancia difusas.




% Poner aquí el resumen.
\cleardoublepage



\thispagestyle{empty}


\begin{center}
{\large\bfseries Predicting Learning in the Nutrition field}\\
\end{center}
\begin{center}
Andrea Morales Garzón\\
\end{center}

%\vspace{0.7cm}
\noindent{\textbf{Keywords}: Word Embedding, Food Computing, Data Alignment, Fuzzy Logic, Natural Language Processing}\\

\vspace{0.7cm}
\noindent{\textbf{Abstract}}\\


Heterogeneous data integration has become an essential task in the Food Computing area and many others. The large volumes of data involved, the hyper-specialized vocabulary, the multicultural factors and the assumptions in food language turn it into a difficult problem to address. In this work, we propose a general method based on predictive language models to map data sources from the textual description of their elements. Specifically, we have developed a computational system that combines a \textit{Word Embedding} model to represent textual data and a mapping procedure that applies different types of distance metrics (syntactic/semantic, crisp/fuzzy) to match them. In order to illustrate its capabilities and the potential of this system, it has been applied to solve the problem of automated adaptation of recipes to food restrictions. Experiments show that this approach is appropriate for solving the problem, particularly when domain-specific models are combined with fuzzy distance metrics.



% \chapter*{}
% \thispagestyle{empty}

% \noindent\rule[-1ex]{\textwidth}{2pt}\\[4.5ex]

% Yo, \textbf{Andrea Morales Garzón}, alumno de la titulación Máster en Ingeniería Informática de la \textbf{Escuela Técnica Superior
% de Ingenierías Informática y de Telecomunicación de la Universidad de Granada}, con DNI 77147632-C, autorizo la
% ubicación de la siguiente copia de mi Trabajo Fin de Máster en la biblioteca del centro para que pueda ser
% consultada por las personas que lo deseen.

% \vspace{6cm}

% \noindent Fdo: Andrea Morales Garzón

% \vspace{2cm}

% \begin{flushright}
% Granada a \today .
% \end{flushright}


% \chapter*{}
% \thispagestyle{empty}

% \noindent\rule[-1ex]{\textwidth}{2pt}\\[4.5ex]

% D. \textbf{Juan Gómez Romero}, Profesor titular del Departamento Ciencias de la Computación e Inteligencia Artificial de la Universidad de Granada.

% \vspace{0.5cm}

% Dña. \textbf{María José Martín Bautista}, Catedrática del Departamento Ciencias de la Computación e Inteligencia Artificial de la Universidad de Granada.


% \vspace{0.5cm}

% \textbf{Informan:}

% \vspace{0.5cm}

% Que el presente trabajo, titulado \textit{\textbf{Aprendizaje predictivo en Nutrición}},
% ha sido realizado bajo su supervisión por \textbf{Andrea Morales Garzón}, y autorizamos la defensa de dicho trabajo ante el tribunal
% que corresponda.

% \vspace{0.5cm}

% Y para que conste, expiden y firman el presente informe en Granada a \today.

% \vspace{1cm}

% \textbf{Los directores:}

% \vspace{5cm}

% \noindent \textbf{Juan Gómez Romero \ \ \ \ \ \ \ \ \ María José Martín Bautista}











\chapter*{Agradecimientos}
\thispagestyle{empty}

       \vspace{1cm}

En primer lugar, quiero expresarle mi gratitud a mis tutores, María José y Juan, por su supervisión y guía a lo largo de este trayecto. Gracias por vuestra experiencia, ejemplo,  dedicación, paciencia y saber hacer. Lo que podría haber quedado en un trabajo de máster, ha resultado ser una experiencia mucho más que enriquecedora en mi formación.\\

Por otra parte, este trabajo tampoco habría sido posible sin el soporte del proyecto europeo Stance4Health. Querría agradecerle a mis superiores en este proyecto el abrirme la vía al, hasta entonces desconocido para mí, mundo del Food Computing y de la Nutrición. Asimismo toda la experiencia que me ha aportado y todas las personas que he conocido. Cada uno de ellos aporta su grano de arena a mi experiencia profesional. \\

En último lugar, a mi madre, por ser un apoyo constante en mi vida en todas sus facetas, incluido el trascurso de este trabajo. Eres mi persona favorita. 

